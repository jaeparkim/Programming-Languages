\chapter{Writing Concurrent Programs}
\label{Writing Concurrent Programs}
This chapter introduces the basic concepts of SALSA
programming, mainly about concurrency coordination. 
Basic knowledge of Java programming is required.

\section{SALSA Language Support for Actor State Modification}
\label{SALSA Language Support for Actor State Modification}
SALSA is a dialect of Java, and it is intended to reuse as many 
features of Java as possible. SALSA actors can contain internal state 
in the form of Java objects or primitive types. However, it is important 
that this internal state must be completely encapsulated, that is, not
shared with other actors. It is also important that the internal state
be serializable. 
\footnote{SALSA, as of version \salsaversion, does not enforce 
state encapsulation and serializability. This discipline needs to be 
followed by programmers, especially for distributed applications.}
   
The following piece of code illustrates how
the internal state is modified, as follows:

%begin{latexonly} 
{\singlespace
\lstinputlisting[]{code/3.3.txt}
}
%end{latexonly} 

\begin{htmlonly}
 \begin{rawhtml} 
  <table border="1" cellpadding="2" cellspacing="0" style="border-collapse: collapse" bordercolor="#111111" id="AutoNumber1">
   <tr><td><pre>
  \end{rawhtml} 
\input{htmlcode/3.3.txt}
 \begin{rawhtml} 
   </pre></td></tr>
  </table>
\end{rawhtml} 
\end{htmlonly} 


\section{Actor Creation}
\label{Actor Creation}
The actor reference is one of the main new primitive 
values of SALSA. There are three approaches to get an 
actor reference: either by actor creation statement, 
the {\tt getReferenceByName()} function, or passed arguments
from messages. This section only concentrates on actor 
creation and reference passing. 

Writing a constructor in SALSA 
programming is similar 
to object construction in Java programming. 
For instance, one can declare the {\tt HelloWorld}
actor as follows:

%begin{latexonly} 
{\singlespace
\lstinputlisting[]{code/3.1.txt}
}
%end{latexonly} 
\begin{htmlonly}
 \begin{rawhtml} 
  <table border="1" cellpadding="2" cellspacing="0" style="border-collapse: collapse" bordercolor="#111111" id="AutoNumber1">
   <tr><td><pre>
  \end{rawhtml} 
\input{htmlcode/3.1.txt}
 \begin{rawhtml} 
   </pre></td></tr>
  </table>
\end{rawhtml} 
\end{htmlonly} 



To create an actor instance and return a reference to 
{\tt myRef}, one can write code as follows:
%begin{latexonly} 
{\singlespace
\lstinputlisting[]{code/3.2.txt}
}
%end{latexonly} 
\begin{htmlonly}
 \begin{rawhtml} 
  <table border="1" cellpadding="2" cellspacing="0" style="border-collapse: collapse" bordercolor="#111111" id="AutoNumber1">
   <tr><td><pre>
  \end{rawhtml} 
\input{htmlcode/3.2.txt}
 \begin{rawhtml} 
   </pre></td></tr>
  </table>
\end{rawhtml} 
\end{htmlonly} 

In SALSA, actor 
references are passed by reference, while object 
variables by value. Objects are passed by value to 
prevent shared memory and preserve encapsulation.

\section{Message Passing}
SALSA uses asynchronous message passing as its primitive form of communication. 
A SALSA\textit{message handler} is similar to a Java method.
Message passing in SALSA is implemented by  
asynchronous message delivery with 
dynamic method invocation. The following example 
shows how an actor sends a message to itself. 
Note that it is not a Java method invocation:
%begin{latexonly} 
{\singlespace
\lstinputlisting[]{code/3.4.txt}
}
%end{latexonly} 
\begin{htmlonly}
 \begin{rawhtml} 
  <table border="1" cellpadding="2" cellspacing="0" style="border-collapse: collapse" bordercolor="#111111" id="AutoNumber1">
   <tr><td><pre>
  \end{rawhtml} 
\input{htmlcode/3.4.txt}
 \begin{rawhtml} 
   </pre></td></tr>
  </table>
\end{rawhtml} 
\end{htmlonly} 


Another type of message passing statement 
requires a target (an actor reference), 
a reserved token {\tt {\textless}-}, 
and a message handler with arguments to be sent. 
For instance, an actor can send a message to 
the standardOutput actor as follows:
%begin{latexonly} 
{\singlespace
\lstinputlisting[]{code/3.5.txt}
}
%end{latexonly}
\begin{htmlonly}
 \begin{rawhtml} 
  <table border="1" cellpadding="2" cellspacing="0" style="border-collapse: collapse" bordercolor="#111111" id="AutoNumber1">
   <tr><td><pre>
  \end{rawhtml} 
\input{htmlcode/3.5.txt}
 \begin{rawhtml} 
   </pre></td></tr>
  </table>
\end{rawhtml} 
\end{htmlonly} 
 
Note that the following expression is illegal 
because it is neither  
a Java method invocation nor a 
message passing expression:
%begin{latexonly} 
{\singlespace
\lstinputlisting[]{code/3.6.txt}
}
%end{latexonly} 
\begin{htmlonly}
 \begin{rawhtml} 
  <table border="1" cellpadding="2" cellspacing="0" style="border-collapse: collapse" bordercolor="#111111" id="AutoNumber1">
   <tr><td><pre>
  \end{rawhtml} 
\input{htmlcode/3.6.txt}
 \begin{rawhtml} 
   </pre></td></tr>
  </table>
\end{rawhtml} 
\end{htmlonly} 

Message Passing in SALSA is by-value 
for objects, and 
by-reference for actors. Any object
passed as an argument is cloned at the moment it 
is sent, and the cloned object is then sent to the target actor.

\section{Coordinating Concurrency}
\label{Coordinating Concurrency}
SALSA provides three approaches to coordinate behaviors of actors: 
\textit{token-passing continuations}, \textit{join blocks}, 
and \textit{first-class continuations}. 

\subsection{Token-Passing Continuations}
\label{Token passing continuations}
Token-passing continuations 
are designed to specify a partial order of message processing.
The token '{\tt @}' is used to group messages 
and assigns the execution order to each of them. 
For instance, the following example 
forces the {\tt standardOutput} actor, a predefined system actor for 
output, to print out "Hello World":
%begin{latexonly} 
{\singlespace
\lstinputlisting[]{code/3.7.txt}
}
%end{latexonly} 
\begin{htmlonly}
 \begin{rawhtml} 
  <table border="1" cellpadding="2" cellspacing="0" style="border-collapse: collapse" bordercolor="#111111" id="AutoNumber1">
   <tr><td><pre>
  \end{rawhtml} 
\input{htmlcode/3.7.txt}
 \begin{rawhtml} 
   </pre></td></tr>
  </table>
\end{rawhtml} 
\end{htmlonly} 

If a programmer uses '{\tt ;}' instead of 
'{\tt @}', SALSA does not guarantee  
that the standardOutput actor will print out "Hello World". It is 
possible to have the result 
"WorldHello ". The following example shows the 
non-deterministic case:

%begin{latexonly} 
{\singlespace
\lstinputlisting[]{code/3.8.txt}
}
%end{latexonly} 
\begin{htmlonly}
 \begin{rawhtml} 
  <table border="1" cellpadding="2" cellspacing="0" style="border-collapse: collapse" bordercolor="#111111" id="AutoNumber1">
   <tr><td><pre>
  \end{rawhtml} 
\input{htmlcode/3.8.txt}
 \begin{rawhtml} 
   </pre></td></tr>
  </table>
\end{rawhtml} 
\end{htmlonly} 

A SALSA message handler can return a value, and 
the value can be accessed through a reserved keyword '{\tt token}', 
specified in one of the arguments of the next grouped message. 
For instance, assuming there exists a user-defined 
message handler, {\tt returnHello()}, which 
returns a string "Hello". The following example 
prints out "Hello" to the standard output:
%begin{latexonly} 
{\singlespace
\lstinputlisting[]{code/3.9.txt}
}
%end{latexonly}
\begin{htmlonly}
 \begin{rawhtml} 
  <table border="1" cellpadding="2" cellspacing="0" style="border-collapse: collapse" bordercolor="#111111" id="AutoNumber1">
   <tr><td><pre>
  \end{rawhtml} 
\input{htmlcode/3.9.txt}
 \begin{rawhtml} 
   </pre></td></tr>
  </table>
\end{rawhtml} 
\end{htmlonly} 
 
Again, assuming another user-defined message handler 
{\tt combineStrings()}
accepts two input Strings and returns a combined string of the 
inputs, the following example prints out "Hello World" 
to the standard ouput:
%begin{latexonly} 
{\singlespace
\lstinputlisting[]{code/3.10.txt}
}
%end{latexonly}
\begin{htmlonly}
 \begin{rawhtml} 
  <table border="1" cellpadding="2" cellspacing="0" style="border-collapse: collapse" bordercolor="#111111" id="AutoNumber1">
   <tr><td><pre>
  \end{rawhtml} 
\input{htmlcode/3.10.txt}
 \begin{rawhtml} 
   </pre></td></tr>
  </table>
\end{rawhtml} 
\end{htmlonly} 
 
Note that the first token refers to the return value of 
{\tt returnHello()}, and the second token refers to that of 
{\tt combineStrings(token, " World")}.

\subsection{Join Blocks}
\label{Join Blocks}
The previous sub-section has illustrated how token-passing 
continuations work in message passing.  
This sub-section introduces join blocks 
which can specify a barrier for parallel processing activities and 
join their results in a subsequent message. A join continuation has a scope (or block) 
starting with "{\tt join\{ }" and ending with "{\tt \}}". 
Every message inside the block must be executed, 
and then the continuation message, following {\tt @}, can be sent. 
For instance, the following example prints 
either "Hello World SALSA" or "WorldHello  SALSA":

%begin{latexonly} 
{\singlespace
\lstinputlisting[]{code/3.11.txt}
}
%end{latexonly}
\begin{htmlonly}
 \begin{rawhtml} 
  <table border="1" cellpadding="2" cellspacing="0" style="border-collapse: collapse" bordercolor="#111111" id="AutoNumber1">
   <tr><td><pre>
  \end{rawhtml} 
\input{htmlcode/3.11.txt}
 \begin{rawhtml} 
   </pre></td></tr>
  </table>
\end{rawhtml} 
\end{htmlonly} 
 
Using the return token of the join block will be explained in 
Chapter~\ref{Advanced Concurrency Coordination}.

\subsection{First-Class Continuations}
\label{First-Class Continuations}
The purpose of first-class continuations is to delegate 
computation to a third party, enabling dynamic replacement or 
expansion of messages grouped by token-passing continuations. 
First-class continuations are very useful for writing recursive 
code. In SALSA, the keyword {\tt currentContinuation} 
is reserved for first-class continuations. To explain the effect 
of first-class continuations, we use two examples to show the difference. In the first example, 
statement 1 prints out "Hello World SALSA":
%begin{latexonly} 
{\singlespace
\lstinputlisting[]{code/3.12.txt}
}
%end{latexonly} 
\begin{htmlonly}
 \begin{rawhtml} 
  <table border="1" cellpadding="2" cellspacing="0" style="border-collapse: collapse" bordercolor="#111111" id="AutoNumber1">
   <tr><td><pre>
  \end{rawhtml} 
\input{htmlcode/3.12.txt}
 \begin{rawhtml} 
   </pre></td></tr>
  </table>
\end{rawhtml} 
\end{htmlonly} 

In the following (the second) example, statement 2 may generate 
a different result from statement 1. It prints out either 
"Hello World SALSA", or "SALSAHello World ". 
%begin{latexonly} 
{\singlespace
\lstinputlisting[]{code/3.13.txt}
}
%end{latexonly} 
\begin{htmlonly}
 \begin{rawhtml} 
  <table border="1" cellpadding="2" cellspacing="0" style="border-collapse: collapse" bordercolor="#111111" id="AutoNumber1">
   <tr><td><pre>
  \end{rawhtml} 
\input{htmlcode/3.13.txt}
 \begin{rawhtml} 
   </pre></td></tr>
  </table>
\end{rawhtml} 
\end{htmlonly} 

The keyword {\tt currentContinuation} has another impact on 
message passing --- the control of execution returns immediately 
after processing it. Any code after it is meaningless. For 
instance, the following piece of code always prints out "Hello World", 
but "SALSA" never gets printed:
%begin{latexonly} 
{\singlespace
\lstinputlisting[]{code/3.14.txt}
}
%end{latexonly} 
\begin{htmlonly}
 \begin{rawhtml} 
  <table border="1" cellpadding="2" cellspacing="0" style="border-collapse: collapse" bordercolor="#111111" id="AutoNumber1">
   <tr><td><pre>
  \end{rawhtml} 
\input{htmlcode/3.14.txt}
 \begin{rawhtml} 
   </pre></td></tr>
  </table>
\end{rawhtml} 
\end{htmlonly} 

\section{Input/Output (I/O) Actor Access}
SALSA provides three kinds of I/O actors to support asynchronous I/O accesses.
One of them is an input service ({\tt standardInput}), and two are 
output services ({\tt standardOutput} and {\tt standardError}).
Since they are actors, only asynchronous accesses are possible.

{\tt standardOutput} provides the following message handlers:
{\singlespace
\begin{itemize}
\item {\tt print(boolean p)}
\item {\tt print(byte p)}
\item {\tt print(char p)}
\item {\tt print(double p)}
\item {\tt print(float p)}
\item {\tt print(int p)}
\item {\tt print(long p)}
\item {\tt print(Object p)}
\item {\tt print(short p)}
\item {\tt println(boolean p)}
\item {\tt println(byte p)}
\item {\tt println(char p)}
\item {\tt println(double p)}
\item {\tt println(float p)}
\item {\tt println(int p)}
\item {\tt println(long p)}
\item {\tt println(Object p)}
\item {\tt println(short p)}
\item {\tt println()}
\end{itemize}
 }
{\tt standardError} provides the following message handlers:
{\singlespace
\begin{itemize}
\item {\tt print(boolean p)}
\item {\tt print(byte p)}
\item {\tt print(char p)}
\item {\tt print(double p)}
\item {\tt print(float p)}
\item {\tt print(int p)}
\item {\tt print(long p)}
\item {\tt print(Object p)}
\item {\tt print(short p)}
\item {\tt println(boolean p)}
\item {\tt println(byte p)}
\item {\tt println(char p)}
\item {\tt println(double p)}
\item {\tt println(float p)}
\item {\tt println(int p)}
\item {\tt println(long p)}
\item {\tt println(Object p)}
\item {\tt println(short p)}
\item {\tt println()}
\end{itemize}
}
{\tt standardInput} provides only one message handler in current SALSA release:
{\singlespace
\begin{itemize}
\item {\tt String readLine()}
\end{itemize}
}

\section{Writing Your First SALSA Program}
\label{Writing Your First SALSA Program}
This section demonstrates how to write, compile, and 
execute your SALSA programs.

\subsection{Steps to Compile and Execute Your SALSA Programs}
\label{Steps to compile and execute your SALSA program}
SALSA hides many of the details involved in developing distributed 
open systems. SALSA programs are preprocessed into Java source code. 
The generated Java code uses a library that supports all the actor's 
primitives --- mainly creation, migration, and communication. Any Java 
compiler can then be used to convert the generated code into Java 
bytecode ready to be executed on any virtual machine 
implementation (see Table~\ref{tbl1}).
 
\begin{table}[top]
\caption{Steps to Compile and Execute Your SALSA Program.}
\label{tbl1}        % \label command must always comes AFTER the caption
\begin{center}
\begin{tabular}{|l|l|l|}
\hline
Step       & What to DO     & Actual Action Taken  \\
\hline\hline
1           & Create a SALSA program:     & Write your SALSA code\\
            & Program.salsa             & \\
\hline
2           & Use the SALSA compiler to   & java salsac.SalsaCompiler \\
            & generate a Java source file:&    Program.salsa\\
            & Program.java              & \\
\hline
3           & Use a Java compiler to      & javac Program.java \\
            & generate the Java bytecode: & \\
            & Program.class             & \\
\hline
4           & Run your program using the  & java Program \\
            & Java Virtual Machine        & \\ 
\hline
\end{tabular}
\end{center}
\end{table}

\subsection{{\tt HelloWorld} example}
The following piece of code is the SALSA version of {\tt HelloWorld}
program:
%begin{latexonly} 
{\singlespace
\lstinputlisting[]{code/HelloWorld.txt}
}
%end{latexonly} 
\begin{htmlonly}
 \begin{rawhtml} 
  <table border="1" cellpadding="2" cellspacing="0" style="border-collapse: collapse" bordercolor="#111111" id="AutoNumber1">
   <tr><td><pre>
  \end{rawhtml} 
\input{htmlcode/HelloWorld.txt}
 \begin{rawhtml} 
   </pre></td></tr>
  </table>
\end{rawhtml} 
\end{htmlonly} 

Let us go step by step through the code of the {\tt HelloWorld.salsa} 
program:

The first line is a comment. SALSA syntax is very similar to Java and you will 
notice it uses the style of Java programming. 
The module keyword is similar to the package keyword in Java. A {\tt module} 
is a collection of related actor behaviors. A {\tt module} can group several 
actor interfaces and behaviors.
Line 4 starts the definition of the {\tt act} message handler. 
In fact, every 
SALSA application must contain the  
following signature if it does have an {\tt act} message handler:

%begin{latexonly} 
{\singlespace
\lstinputlisting[]{code/3.15.txt}
}
%end{latexonly} 
\begin{htmlonly}
 \begin{rawhtml} 
  <table border="1" cellpadding="2" cellspacing="0" style="border-collapse: collapse" bordercolor="#111111" id="AutoNumber1">
   <tr><td><pre>
  \end{rawhtml} 
\input{htmlcode/3.15.txt}
 \begin{rawhtml} 
   </pre></td></tr>
  </table>
\end{rawhtml} 
\end{htmlonly}
 
When a SALSA application is executed, an actor with the specified behavior 
is created and an {\tt act} message is sent to it by the run-time environment. 
The {\tt act} message is used as a bootstrapping mechanism for SALSA programs. 
It is analogous to the Java {\tt main} method invocation.

In lines 5 and 6, two messages are sent to the {\tt standardOutput} actor.
The arrow ({\tt \textless-}) indicates message sending to an actor 
(in this case, the {\tt standardOutput} actor). To guarantee that the 
messages are received 
in the same order they were sent, the {\tt @} sign is used to enforce  
the second message to be sent only after the first message 
has been processed. This is referred to as \textit {a token-passing continuation}
(see Section \ref{Token passing continuations}). 

\subsection{Compiling and Running {\tt HelloWorld}}
\begin{itemize}

\item  Download the latest version of SALSA. You will find the 
latest release in this URL: http://wcl.cs.rpi.edu/salsa/

\item Create a directory called {\tt examples} and save the {\tt HelloWorld.salsa} 
program inside it. You can use any simple text editor or Java editor 
to write your SALSA programs. Note that SALSA modules are similar to 
Java packages. So you have to follow the directory structure conventions when 
working with modules as you do when working with packages in Java.

\item Compile the SALSA source file into a Java source file using the 
SALSA compiler. It is recommended to include the SALSA JAR file in your 
class path. Alternatively you can use {\tt -cp} to specify its path in the 
command line. If you are using MS Windows use semi-colon ({\tt ;}) as a 
class path delimiter, if you are using just about anything else, 
use colon ({\tt :}). For example:

\textbf{java -cp salsa{\textless}version{\textgreater}.jar:. 
salsac.SalsaCompiler examples/*.salsa}

\item Use any Java compiler to compile the generated Java file. 
Make sure to specify the SALSA class path using {\tt -classpath} if 
you have not included it already in your path.

\textbf{javac -classpath salsa{\textless}version{\textgreater}.jar:. examples/*.java}

\item Execute your program

\textbf{java -cp salsa{\textless}version{\textgreater}.jar:. examples.HelloWorld}

\end{itemize}
