\chapter{Writing Distributed Programs}
\label{Writing Distributed Programs}
Distributed SALSA programming involves universal naming, theaters,
service actors, migration, and concurrency control. This chapter
introduces how to write and run a distributed SALSA program.

\section{Worldwide Computing Model}
\label{Worldwide Computing Model}
Worldwide computing is an emerging discipline with the goal of 
turning the web into a unified distributed computing infrastructure. 
Worldwide computing tries 
to harness underutilized resources in the web by providing to various 
internet users, a unified interface that allows them to distribute their 
computation in a global fashion without having to worry about where resources 
are located and what platforms are being used. Worldwide computing is based on 
the actor model of concurrent computation and implements several strategies for 
distributed computation such as universal naming, message passing, and migration. 
This section introduces the worldwide computing model 
and how it is supported by SALSA.

\subsection{World-Wide Computer (WWC) Architecture}
The World-Wide Computer (WWC) is a set of virtual machines, or \textit{theaters} that host 
one to many concurrently running universal actors. Theaters provide a layer 
beneath actors for message passing, remote communication, and migration. Every theater 
consists of a RMSP (Remote Message Sending Protocol) server, a local cache 
that maps between actors' names and their current locations, a registry that 
maps local actor names to their references, and a run-time environment. 
The RMSP server runs forever listening for incoming requests from remote 
actors and starting multiples threads to handle incoming requests simultaneously.

The WWC consists of the following key components:
\begin{itemize}
\item Universal naming service
\item Run-time environment
\item Remote communication protocol
\item Migration support
\item Actor garbage collection
\end{itemize}


\subsection{Universal Naming - UAN and UAL}
Universal naming allows actors to become \textit{universal actors}. 
Universal actors have the ability to migrate, while
anonymous actors do not. \textit{Service actors} are a special kind of 
universal actors,
which grant universal access privileges to every actor
and never get collected by the actor garbage collection mechanism. 

Every universal actor has a \textit{Universal Actor Name (UAN)}, 
and a \textit{Universal Actor Locator (UAL)}. The UAN is a unique name 
that identifies the actor throughout its lifetime. The UAL 
represents the location where the actor 
is currently running. While the UAN never changes throughout 
the lifetime of a universal actor, its UAL changes as it migrates 
from one location to another. The format 
of UAN and UAL follow the URI syntax. They are similar in format to a 
URL (see Table~\ref{tbl2}).

\begin{table}[top]
\caption{UAL and UAN.}
\label{tbl2}        
\begin{center}
\begin{tabular}{|l|l|}
\hline
Type       & Example\\
\hline\hline
URL & http://wcl.cs.rpi.edu/salsa/ \\
\hline
UAN & uan://io.wcl.cs.rpi.edu:3000/myName \\
\hline
UAL & rmsp://io.wcl.cs.rpi.edu:4000/myLocator \\
\hline
\end{tabular}
\end{center}
\end{table}

\paragraph{UAN:} The first item of the UAN specifies the name of the protocol used; the second 
item specifies the name and port number of the machine where the Naming Server 
resides. This name is usually a name that can be decoded by a domain name server. 
You can also use the IP of the machine. But this is a practice that should be avoided. 
The last item specifies the relative name of the actor. If a port number is not specified, the 
default port number (3030) for the name server is used. 

\paragraph{UAL:} The first item specifies the protocol used for remote message sending. The second 
item indicates the theater's machine name and port number. If a port number is not specified, the 
default port number (4040) for the theater is used.  The last part specifies the relative locator name 
of the actor in the given 
theater.

SALSA's universal naming scheme has been designed in such a way to satisfy the following requirements:
\begin{itemize}
\item Platform independence: names appear coherent on all nodes independent of the 
underlying architecture.
\item Scalability of name space management
\item Transparent actor migration
\item Openness by allowing unanticipated actor reference creation and protocols that 
provide access through names
\item Both human and computer readability.
\end{itemize}

\section{How SALSA Supports the Worldwide Computing Model}
\label{How SALSA Supports the Worldwide Computing Model}
This section demonstrates how to write a distributed
SALSA program and run it in the World-Wide Computer run-time environment.

\subsection{Universal Actor Creation}
A universal actor can be created at any desired theater by specifying its 
UAN and UAL~\footnote{Remember to start the naming server if
using UANs in the computation.}.
For instance, one can create a universal actor at current 
host as follows:
%begin{latexonly} 
{\singlespace
\lstinputlisting[]{code/5.1.txt}
}
%end{latexonly} 
\begin{htmlonly}
 \begin{rawhtml} 
  <table border="1" cellpadding="2" cellspacing="0" style="border-collapse: collapse" bordercolor="#111111" id="AutoNumber1">
   <tr><td><pre>
  \end{rawhtml} 
\input{htmlcode/5.1.txt}
 \begin{rawhtml} 
   </pre></td></tr>
  </table>
\end{rawhtml} 
\end{htmlonly}
 
A universal actor can be created at a remote theater, hosted at host1:4040, 
by the following statement:
%begin{latexonly} 
{\singlespace
\lstinputlisting[]{code/5.2.txt}
}
%end{latexonly} 
\begin{htmlonly}
 \begin{rawhtml} 
  <table border="1" cellpadding="2" cellspacing="0" style="border-collapse: collapse" bordercolor="#111111" id="AutoNumber1">
   <tr><td><pre>
  \end{rawhtml} 
\input{htmlcode/5.2.txt}
 \begin{rawhtml} 
   </pre></td></tr>
  </table>
\end{rawhtml} 
\end{htmlonly}
 
An anonymous actor can be created as follows:
%begin{latexonly} 
{\singlespace
\lstinputlisting[]{code/5.3.txt}
}
%end{latexonly} 
\begin{htmlonly}
 \begin{rawhtml} 
  <table border="1" cellpadding="2" cellspacing="0" style="border-collapse: collapse" bordercolor="#111111" id="AutoNumber1">
   <tr><td><pre>
  \end{rawhtml} 
\input{htmlcode/5.3.txt}
 \begin{rawhtml} 
   </pre></td></tr>
  </table>
\end{rawhtml} 
\end{htmlonly}
 
Notice that an anonymous actor cannot migrate.

\subsection{Referencing actors}

Actor references can be used as the target of message sending expressions 
or as arguments of messages. 
There are three ways to get actor references. Two of them, the return 
value of actor creation and references from messages, are available 
in both distributed and concurrent SALSA programming. The last one, 
{\tt getReferenceByName()}, is an explicit approach that translates a 
string representing a UAN into a reference. In SALSA, only references 
to service actors (see Section \ref{ServiceActor}) should be obtained 
using this function. Otherwise, SALSA does not guarantee the safety property 
of actor garbage collection, which means one can get a phantom reference 
(a reference pointing to nothing). The way to get a reference by 
{\tt getReferenceByName()} is shown as follows:
%begin{latexonly} 
{\singlespace
\lstinputlisting[]{code/5.4.txt}
}
%end{latexonly} 
\begin{htmlonly}
 \begin{rawhtml} 
  <table border="1" cellpadding="2" cellspacing="0" style="border-collapse: collapse" bordercolor="#111111" id="AutoNumber1">
   <tr><td><pre>
  \end{rawhtml} 
\input{htmlcode/5.4.txt}
 \begin{rawhtml} 
   </pre></td></tr>
  </table>
\end{rawhtml} 
\end{htmlonly}
 
Sometimes an actor wants to know its name or location. An actor can get its 
UAL (location) by the function {\tt getUAL()} and UAN (universal name) by 
{\tt getUAN()}. For example:
%begin{latexonly} 
{\singlespace
\lstinputlisting[]{code/5.5.txt}
}
%end{latexonly} 
\begin{htmlonly}
 \begin{rawhtml} 
  <table border="1" cellpadding="2" cellspacing="0" style="border-collapse: collapse" bordercolor="#111111" id="AutoNumber1">
   <tr><td><pre>
  \end{rawhtml} 
\input{htmlcode/5.5.txt}
 \begin{rawhtml} 
   </pre></td></tr>
  </table>
\end{rawhtml} 
\end{htmlonly}


\subsection{Migration}
As mentioned before, only universal actors can migrate. 
Sending the message {\tt migrate({\textless}ual{\textgreater})} to an universal actor causes it to migrate 
seamlessly to the designated location. Its UAL will be changed and the 
naming service will be notified to update its entry. 

The following example defines the behavior {\tt MigrateSelf}, 
that migrates the {\tt MigrateSelf} actor to location UAL1 and 
then to UAL2. The {\tt migrate} message takes as argument a 
string specifying the target UAL or it can take the object 
{\tt new UAL("UAL string")}.
%begin{latexonly} 
{\singlespace
\lstinputlisting[]{code/MigrateSelf.salsa.txt}
}
%end{latexonly} 
\begin{htmlonly}
 \begin{rawhtml} 
  <table border="1" cellpadding="2" cellspacing="0" style="border-collapse: collapse" bordercolor="#111111" id="AutoNumber1">
   <tr><td><pre>
  \end{rawhtml} 
\input{htmlcode/MigrateSelf.salsa.txt}
 \begin{rawhtml} 
   </pre></td></tr>
  </table>
\end{rawhtml} 
\end{htmlonly}
 
\subsection{Actors as Network Service} \label{ServiceActor}
There are many kinds of practical distributed applications: some are 
designed for scientific computation, which
may produce a lot of temporary actors for parallel processing; 
some are developed for network services, such as a web server, a web search 
engine, etc. Useless actors should be reclaimed for memory reuse, 
while service-oriented actors must remain available under any circumstance.

The most important reason for reclamation of useless actors is to avoid 
memory leakage. For example, after running the {\tt HelloWorld} actor 
(shown in Section~\ref{Writing Your First SALSA Program}) in 
the World-Wide Computer, the World-Wide 
Computer must be able to reclaim this actor after it prints out "Hello World". 
Reclamation of actors is formally named \textit{actor garbage collection}.

Reclamation of useless actors introduces a new problem: how to support 
non-collectable service-oriented actors at the language level. This is important 
because a service-oriented actor cannot be reclaimed even if it is idle. For 
instance, a web service should always wait for requests. Reclamation of an idle 
service is wrong.

A service written by C or Java programming languages uses infinite loops to listen for
requests. 
A SALSA service should not use this approach because loops inside a message handler preclude an actor
from executing messages in its message box. The way SALSA keeps a service actor alive is by 
specifying it at the language level - a SALSA \textit{service actor} must implement the 
interface {\tt ActorService} to 
tell the actor garbage collector not to collect it.

The following example illustrates how a service actor is implemented 
 in SALSA. The example implements a 
simple address book service. The {\tt AddressBook} actor provides the functionality 
of creating new {\textless}name, email{\textgreater} entities, and 
responding to end users' requests. The example defines the {\tt AddUser} 
message handler which adds new entries 
in the database.  The example also defines the {\tt GetEmail} message handler
which returns an email string providing the user name.
%begin{latexonly} 
{\singlespace
\lstinputlisting[]{code/AddressBook.salsa.txt}
}
%end{latexonly} 
\begin{htmlonly}
 \begin{rawhtml} 
  <table border="1" cellpadding="2" cellspacing="0" style="border-collapse: collapse" bordercolor="#111111" id="AutoNumber1">
   <tr><td><pre>
  \end{rawhtml} 
\input{htmlcode/AddressBook.salsa.txt}
 \begin{rawhtml} 
   </pre></td></tr>
  </table>
\end{rawhtml} 
\end{htmlonly}
 
The {\tt AddressBook} actor is bound to the UAN and UAL pair specified in the command line. 
This will result in placing the {\tt AddressBook} actor in the designated 
location and notifying the naming service.

To be able to contact the {\tt AddressBook} actor, a client actor 
first needs to get the remote reference of the service. The only 
way to get the reference is by the message handler 
{\tt getReferenceByName()}. The example we are going to demonstrate is 
the {\tt AddUser} actor, which communicates with the {\tt AddressBook} actor 
to add new entries. Note that the {\tt AddUser} actor can be started anywhere on the Internet.
%begin{latexonly} 
{\singlespace
\lstinputlisting[]{code/AddUser.salsa.txt}
}
%end{latexonly} 
\begin{htmlonly}
 \begin{rawhtml} 
  <table border="1" cellpadding="2" cellspacing="0" style="border-collapse: collapse" bordercolor="#111111" id="AutoNumber1">
   <tr><td><pre>
  \end{rawhtml} 
\input{htmlcode/AddUser.salsa.txt}
 \begin{rawhtml} 
   </pre></td></tr>
  </table>
\end{rawhtml} 
\end{htmlonly}

\section{Run-time Support for WWC Application}
\label{Run-time Support for WWC Application}
The section demonstrates how to start the naming service and theaters.

\subsection{Universal Naming Service}
The UANP is a protocol that defines how to interact with the WWC naming service. 
Similar to HTTP, UANP is text-based and defines methods that allow actors' names 
lookup, updates, and deletions. UANP operates over TCP connections, usually the port 3030. 
This port can be overwritten by another port number.

Every theater maintains a local registry where actors' locations are cached for faster 
future access. One can start a naming service as follows:

\textbf{java -cp salsa{\textless}version{\textgreater}.jar:. wwc.naming.WWCNamingServer}

The above command starts a naming service on the default port 3030. You can specify another 
port as follows:

\textbf{java -cp salsa{\textless}version{\textgreater}.jar:. wwc.naming.WWCNamingServer -p 1256}

\subsection{Theaters}

One can start a theater as follows:

\textbf{java -cp salsa{\textless}version{\textgreater}.jar:. wwc.messaging.Theater}

The above command starts a theater on the default RMSP port 4040. You can specify 
another port as follows:

\textbf{java -cp salsa{\textless}version{\textgreater}.jar:. wwc.messaging.Theater 4060}


\subsection{Running an Application}
Whenever a WWC application is executed, a theater is dynamically created
to host the bootstrapping actor of the application and a random port is assigned
to the dynamically created theater. A dynamically created theater will be destroyed
if no application actor is hosted at it and no incoming message will be delivered
to the theater.

{\sloppy
Now let us consider a WWC application example.
Assuming a theater is running at {\tt host1:4040}, and a naming 
service at {\tt host2:5555}. One can run the {\tt HelloWorld} 
example {shown in Section~\ref{Writing Your First SALSA Program}) 
at {\tt host1:4040} as follows:

\textbf{java  -cp salsa{\textless}version{\textgreater}.jar:. 
 -Duan=uan://host2:5555/myhelloworld 
 -Dual=rmsp://host1:4040/myaddr examples.HelloWorld}
}

As one can see, the standard output of host1 displays "Hello World". One may also find that
the application does not terminate. In fact, the reason for non-termination at the original 
host is that the application creates a theater and the theater joins the World-Wide Computer 
environment. Formally speaking, the application does terminate but the host to begin with 
becomes a part of the World-Wide Computer.



