\chapter{Actor Garbage Collection}\label{actorGarbageCollection}
Actor garbage collection is the mechanism to reclaim useless
actors. A system eventually fails because of memory leakage, 
resulting from uncollected garbage actors. Manual actor garbage collection 
is error-prone and always reduces 
programmers' efficiency.
Manual garbage collection can solve this problem if an application does not require a
lot of dynamic memory allocation operations. As the size of the application becomes
larger and more complex, automatic garbage collection becomes preferable. There
are two reasons. Firstly, manual garbage collection is error-prone and thus causes
memory security issues. Secondly, manual garbage collection cuts against high-level
programming. From the perspective of software engineering, people should focus on
the development of functionalities, not on irrelevant concerns. Currently, garbage
collection usually refers to automatic garbage collection.

Many object-oriented programming languages support automatic 
garbage collection, such as Smalltalk, Scheme, Java, etc.
Unfortunately, these garbage collection algorithms cannot be 
used for actor garbage collection directly, because each actor 
has a thread of control 
encapsulated in it. An actor can create or delete 
references to other actors, while an object cannot. The thread of 
control results in the essential difference of
actor garbage collection and object garbage collection. 


\section{Actor Garbage Definition}
The definition of actor garbage relates to meaningful computation. 
Meaningful computation is 
defined as having the ability to communicate with any of the \emph{root actors},
that is, to access any resource or public service. The widely used 
definition of live actors is described in \cite{kafura-actorGCdef-91}.
Conceptually, an actor is live if it is a root or it can either potentially: 
1) receive messages from the root actors or 2) send messages to the root 
actors. The set of actor garbage is then defined as the complement of the set of live 
actors. To formally describe our new actor GC model, we introduce the
following definitions:
\begin{itemize}
\item {\bf Blocked actor}:
An actor is blocked if it has no pending messages in 
its message box, nor any message being processed. Otherwise
it is unblocked.
\item {\bf Reference}:
A reference indicates an address 
of an actor. Actor $A$ can only send messages to Actor $B$ if $A$ 
has a reference pointing to $B$.
\item {\bf Inverse reference}:
An inverse reference is a conceptual reference
in the counter-direction of an existing reference. 
\item {\bf Acquaintance}:
Let Actor $A$ have a reference pointing to Actor $B$.
$B$ is an acquaintance of $A$, and $A$ is
an inverse acquaintance of $B$.
\item {\bf Root actor}:
An actor is a root actor if it encapsulates a resource, or 
if it is a public service --- such as I/O devices, web services, and 
databases.
\end{itemize}

The original definition of live actors
is denotational because it uses the concept of {\textquotedblleft}potential{\textquotedblright} message 
delivery and reception. To make it more operational, we use 
the term {\textquotedblleft}\emph{potentially live}{\textquotedblright} \cite{dick96} to define live actors.
\begin{itemize}
\item {{\bf Potentially live actors}: 
\begin{itemize}
\item Every unblocked actor and root actor is potentially live. 
\item Every acquaintance of a potentially live actor 
  is potentially live. 
\end{itemize}}
\item {{\bf Live actors}:
\begin{itemize}
\item A root actor is live. 
\item Every acquaintance of a live actor is live.
\item Every potentially live, inverse acquaintance of a live 
  actor is live.
\end{itemize}}
\end{itemize}

\section{Actor Garbage Collection in SALSA}
This section introduces the assumptions and the actor garbage collection mechanism used in SALSA.

\subsection{The Live Unblocked Actor Principle} \label{The Live Unblocked Actor Principle}
In actor-oriented programming languages, an actor must be able to access 
resources which are encapsulated in service actors. To access a resource, an actor
requires a reference to it. This implies that actors keep persistent references to
some special service actors --- such as the file system service and the standard
output service. Furthermore, an actor can explicitly create references to public
services. For instance, an actor can dynamically convert a string into a reference to
communicate with a service actor, analogous to accessing a web service by a web
browser using a URL.

Without program analysis techniques,
the ability of an actor to access resources provided by 
an actor-oriented programming language implies 
explicit reference creation to access service actors. The
ability to access local service actors (e.g. the standard output) and explicit 
reference creation to public service actors make the statement true: \emph{every
actor has persistent references to root actors}. This statement is important
 because it changes the meaning of actor GC, making actor GC similar to passive
object GC. It leads to the \emph{live unblocked actor principle}, 
which says every unblocked actor is live. Since
each unblocked actor is: 1) an inverse acquaintance of the root actors and 2)
defined as potentially live, it is live according to the definition of actor GC.
With the live unblocked actor principle, every unblocked actor can be viewed
as a root. Liveness of blocked actors depends on the transitive reachability from
unblocked actors and root actors. If a blocked actor is transitively reachable
from an unblocked actor or a root actor, it is defined as potentially live. With
persistent root references, such potentially live, blocked actors are live because
they are inverse acquaintances of some root actors.

Notice that the live unblocked actor principle may not be true 
if considering resource access restrictions. This implies that
different security models may result in different sets of actor garbage,
given the same actor system.
At the programming language level, SALSA assumes the live unblocked actor principle.  

\subsection{Local Actor Garbage Collection}
The difficulty of local actor garbage collection is to get a consistent 
global state and minimize the penalty 
of actor garbage collection. The easiest approach for actor garbage 
collection is to stop the world: no computation is allowed during actor 
garbage collection. There are two major drawbacks of this approach,
the waiting time for message clearance and 
parallelism degrading 
(only the garbage collector is running and all actors are waiting).

A better solution for local actor garbage collection is 
to use a snapshot-based algorithm \cite{wang-varela-snapshot-tr-2006}, which can improve parallelism 
and does not require waiting time for message clearance.
SALSA uses a snapshot based algorithm, together with the SALSA 
garbage detection protocol (the pseudo-root approach) \cite{wang06pr}, 
in order to get a meaningful global
state. A SALSA local garbage collector
uses the meaningful global snapshot to identify garbage actors.

One can observe the behavior of the local actor garbage collector by specifying
the run-time environment options \textit{-Dgcverbose -Dnodie} as follows:  

{\sloppy
\textbf{java -cp salsa{\textless}version{\textgreater}.jar:. -Dgcverbose -Dnodie examples.HelloWorld}
}

To start a theater without running the local actor garbage collection, one can
use the option \textit{-Dnogc} as follows:

\textbf{java -cp salsa{\textless}version{\textgreater}.jar:. -Dnogc examples.HelloWorld}


\subsection{Optional Distributed Garbage Collection Service}
Distributed actor garbage collection is much more complicated because of
the mobility of actors, the difficulty of recording a meaningful global 
state of the distributed system, and the independent execution of the 
distributed and local garbage collection. The SALSA garbage detection 
protocol and the local actor garbage collectors help simplify the problem
--- they can handle acyclic distributed garbage and all local garbage. 

The current SALSA distributed actor garbage collector is implemented as 
a logically centralized service. It is an optional service. When it is 
triggered to manage several hosts, it coordinates the local
collectors to get a meaningful global snapshot. Actors referenced by those 
outside the selected
hosts are never collected. The task of identifying garbage is 
done in the logically centralized service. Once garbage is identified, a garbage
list is then sent to every participating host.

A distributed garbage collection service collects garbage for selected hosts (theaters).
It collects distributed garbage providing a partial view of the system.
SALSA also supports hierarchical actor garbage collection.
To build the garbage collection hierarchy, each distributed garbage collection service
requires its parent (at most one) and its children.  
The usage of the distributed garbage collection service is shown as follows:

\textbf{java -cp salsa{\textless}version{\textgreater}.jar:. gc.serverGC.SServerPRID 
{\textless}n{\textgreater} {\textless}parent{\textgreater}
{\textless}child1 or host1{\textgreater} {\textless}child2 or host2{\textgreater} ......}

$n$ specifies that the service should be activated every $n$ seconds.
$n=-1$ means that the service only executes once and then terminates.
{\tt parent} specifies the address of the parent service, with the format 
{\tt {\textless}ip{\textgreater}:{\textless}port{\textgreater}}.
Invalid format indicates that the service is the root service.
{\tt {\textless}child1 or host1{\textgreater}} indicates
the address of the child service or the target theater, 
with the format {\tt {\textless}ip{\textgreater}:{\textless}port{\textgreater}}.
Notice that a theater is always a leaf.  

To run the distributed garbage collection once, 
one can use the command as follows:

\textbf{java -cp salsa{\textless}version{\textgreater}.jar:. gc.serverGC.SServerPRID -1 x
{\textless}host1{\textgreater} {\textless}host2{\textgreater} ......}

To run it every $40$ seconds, use:

\textbf{java -cp salsa{\textless}version{\textgreater}.jar:. java gc.serverGC.SServerPRID  
{\textless}40{\textgreater} x {\textless}host1{\textgreater} {\textless}host2{\textgreater} .....}

Note that n should be large enough. The default is $n=20$.

