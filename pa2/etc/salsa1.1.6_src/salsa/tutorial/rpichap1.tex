%%%%%%%%%%%%%%%%%%%%%%%%%%%%%%%%%%%%%%%%%%%%%%%%%%%%%%%%%%%%%%%%%%% 
%                                                                 %
%                            CHAPTER ONE                          %
%                                                                 %
%%%%%%%%%%%%%%%%%%%%%%%%%%%%%%%%%%%%%%%%%%%%%%%%%%%%%%%%%%%%%%%%%%% 
 
\chapter{Introduction}\label{Introduction}

\section{The SALSA Distributed Programming Language}
\label{The SALSA Distributed Programming Language}
With the emergence of Internet and mobile computing, a wide range 
of Internet applications have placed new demands and challenges 
such as openness, portability, ability to adapt quickly to changing 
execution environments, and highly dynamic reconfiguration. Current 
programming languages and systems lack support for dynamic 
reconfiguration of applications, where application entities get
moved to different processing nodes at run-time.

Java has provided a lot of support to dynamic web content through 
applets, network class loading, bytecode verification, security, 
and multi-platform compatibility. Moreover, Java is a good framework 
for distributed Internet programming because of its standardized 
representation of objects and serialization support. 
Some of the important libraries that provide support for Internet 
computing are: {\tt java.rmi} for remote method invocation, 
{\tt java.reflection} for run-time introspection, {\tt java.io} for 
serialization, and {\tt java.net} for sockets, datagrams, and URLs.

SALSA (Simple Actor Language, System and Architecture) 
\cite{varela-agha-salsa-oopsla-2001} is an 
actor-oriented programming language designed and implemented to 
introduce the benefits of the actor model while keeping the advantages 
of object-oriented programming. Abstractions include active objects, 
asynchronous message passing, universal naming, migration, and advanced 
coordination constructs for concurrency. SALSA is pre-processed into 
Java and hence preserves many of Java's useful object oriented concepts- 
mainly, encapsulation, inheritance, and polymorphism. SALSA abstractions 
enable the development of dynamically reconfigurable applications. A SALSA
program consists of universal actors that can ne migrated to distributed 
nodes at run-time.

\section{Structure of the Tutorial}\label{Structure of the Tutorial}
This tutorial covers basic concepts of SALSA and illustrates its concurrency 
and distribution models through several examples. Chapter \ref{Actor-Oriented Programming} introduces the 
actor model and how SALSA supports it. Chapter \ref{Writing Concurrent Programs} introduces concurrent 
programming in SALSA, including token-passing continuations, join 
blocks, and first-class continuations.Chapter \ref{Writing Distributed Programs} discusses SALSA's 
support for distributed computing including asynchronous message sending, 
universal naming, and migration. Chapter \ref{Advanced Concurrency Coordination} introduces several advanced 
coordination constructs and how they can be coded in SALSA. 
Chapter \ref{actorGarbageCollection} defines actor garbage and
explains how automatic actor garbage collection works in SALSA.
Appendix \ref{nameserverop} introduces how to specify name server options and how to run applications
with different system properties. Appendix \ref{DebuggingTips} provides debugging tips for 
SALSA programs. Appendix \ref{LearningSALSAEx} provides brief descriptions of 
SALSA example programs. Appendix \ref{GRAMMAR} lists the SALSA grammar.

