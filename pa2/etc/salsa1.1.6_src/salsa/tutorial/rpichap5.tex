\chapter{Advanced Concurrency Coordination}
\label{Advanced Concurrency Coordination}

This chapter introduces advanced constructs for coordination by using named tokens, 
join block continuations, and message properties. 

\section{Named Tokens}
\label{Named Tokens}
Chapter~\ref{Writing Concurrent Programs} has introduced token-passing 
continuations with the reserved keyword
{\tt token}. In this section, 
we will focus on the other type of continuations, 
the \textit{named tokens}.

In SALSA, the return value of the asynchronous message can be declared 
as a variable with type {\tt token}. The variable is 
called a \textit{named token}. Named tokens are designed to more flexibly tell 
the SALSA run-time environment how to continue certain computations.
For example, a token-passing continuation statement can be re-written by name token continuations:
%begin{latexonly} 
{\singlespace
\lstinputlisting[]{code/tokencont2nametoken.salsa.txt}
}
%end{latexonly} 
\begin{htmlonly}
 \begin{rawhtml} 
  <table border="1" cellpadding="2" cellspacing="0" style="border-collapse: collapse" bordercolor="#111111" id="AutoNumber1">
   <tr><td><pre>
  \end{rawhtml} 
\input{htmlcode/tokencont2nametoken.salsa.txt}
 \begin{rawhtml} 
   </pre></td></tr>
  </table>
\end{rawhtml} 
\end{htmlonly}

Name tokens can be used to construct a non-linear partial order for computation,
which cannot be expressed by token-passing continuations.
The following example cannot be re-written by token-passing continuations:
%begin{latexonly} 
{\singlespace
\lstinputlisting[]{code/nonlinearCont.salsa.txt}
}
%end{latexonly} 
\begin{htmlonly}
 \begin{rawhtml} 
  <table border="1" cellpadding="2" cellspacing="0" style="border-collapse: collapse" bordercolor="#111111" id="AutoNumber1">
   <tr><td><pre>
  \end{rawhtml} 
\input{htmlcode/nonlinearCont.salsa.txt}
 \begin{rawhtml} 
   </pre></td></tr>
  </table>
\end{rawhtml} 
\end{htmlonly}

The following example uses name tokens to implement the Fibonacci number application:
%begin{latexonly} 
{\singlespace
\lstinputlisting[]{code/Fibonacci.salsa.txt}
}
%end{latexonly} 
\begin{htmlonly}
 \begin{rawhtml} 
  <table border="1" cellpadding="2" cellspacing="0" style="border-collapse: collapse" bordercolor="#111111" id="AutoNumber1">
   <tr><td><pre>
  \end{rawhtml} 
\input{htmlcode/Fibonacci.salsa.txt}
 \begin{rawhtml} 
   </pre></td></tr>
  </table>
\end{rawhtml} 
\end{htmlonly}
 

Named tokens may be assigned to non-primitive type values, 
message sending expressions, or other named tokens. 
Examples are shown as follows:
%begin{latexonly} 
{\singlespace
\lstinputlisting[]{code/6.1.txt}
}
%end{latexonly} 
\begin{htmlonly}
 \begin{rawhtml} 
  <table border="1" cellpadding="2" cellspacing="0" style="border-collapse: collapse" bordercolor="#111111" id="AutoNumber1">
   <tr><td><pre>
  \end{rawhtml} 
\input{htmlcode/6.1.txt}
 \begin{rawhtml} 
   </pre></td></tr>
  </table>
\end{rawhtml} 
\end{htmlonly}

The following example shows how to use named tokens.
Lines 1-2 are equivalent to lines 3-5, and
lines 1-2 uses a few token declarations, as follows: 
%begin{latexonly} 
{\singlespace
\lstinputlisting[]{code/6.2.txt}
}
%end{latexonly} 
\begin{htmlonly}
 \begin{rawhtml} 
  <table border="1" cellpadding="2" cellspacing="0" style="border-collapse: collapse" bordercolor="#111111" id="AutoNumber1">
   <tr><td><pre>
  \end{rawhtml} 
\input{htmlcode/6.2.txt}
 \begin{rawhtml} 
   </pre></td></tr>
  </table>
\end{rawhtml} 
\end{htmlonly}

The following example demonstrates how named tokens are used in loops, as follows:
%begin{latexonly} 
{\singlespace
\lstinputlisting[]{code/6.3.txt}
}
%end{latexonly} 
\begin{htmlonly}
 \begin{rawhtml} 
  <table border="1" cellpadding="2" cellspacing="0" style="border-collapse: collapse" bordercolor="#111111" id="AutoNumber1">
   <tr><td><pre>
  \end{rawhtml} 
\input{htmlcode/6.3.txt}
 \begin{rawhtml} 
   </pre></td></tr>
  </table>
\end{rawhtml} 
\end{htmlonly}

The above example is equivalent to the following example:
%begin{latexonly} 
{\singlespace
\lstinputlisting[]{code/6.4.txt}
}
%end{latexonly} 
\begin{htmlonly}
 \begin{rawhtml} 
  <table border="1" cellpadding="2" cellspacing="0" style="border-collapse: collapse" bordercolor="#111111" id="AutoNumber1">
   <tr><td><pre>
  \end{rawhtml} 
\input{htmlcode/6.4.txt}
 \begin{rawhtml} 
   </pre></td></tr>
  </table>
\end{rawhtml} 
\end{htmlonly}

To learn more about named tokens, we use the following example to illustrate 
how the named token declaration works and to prevent confusion:
%begin{latexonly} 
{\singlespace
\lstinputlisting[]{code/6.5.txt}
}
%end{latexonly} 
\begin{htmlonly}
 \begin{rawhtml} 
  <table border="1" cellpadding="2" cellspacing="0" style="border-collapse: collapse" bordercolor="#111111" id="AutoNumber1">
   <tr><td><pre>
  \end{rawhtml} 
\input{htmlcode/6.5.txt}
 \begin{rawhtml} 
   </pre></td></tr>
  </table>
\end{rawhtml} 
\end{htmlonly}

As the token is updated as soon as the code is processed, this leads to some interesting 
occurrences. In the {\tt for} loop on lines 3-7, for each iteration of the loop, the value of 
{\tt token x} in {\tt b{\textless}-m2} and {\tt c{\textless}-m3} is the same. However, 
the value of {\tt token x} in {\tt d{\textless}-m4} is the token returned by 
{\tt c{\textless}-m3}, and thus equal to the value of {\tt token x} in the message 
sends on lines 4 and 5 in the next iteration of the loop.

\section{Join Block Continuations}

Chapter~\ref{Writing Concurrent Programs} skips some important issues of join 
blocks. In this section, 
we are going to introduce how to use the return values of statements inside a join block and how to 
implement the join block reception handler.

A join block always returns \textit{an object array} if it joins 
several messages to a reserved keyword {\tt token}, or a named token. 
If those message handlers to be joined do not return
({\tt void} type return), or the return values are ignored, the join block 
functions like a barrier for parallel message processing.

The named token can be applied to the join block as follows:
%begin{latexonly} 
{\singlespace
\lstinputlisting[]{code/6.6.txt}
}
%end{latexonly} 
\begin{htmlonly}
 \begin{rawhtml} 
  <table border="1" cellpadding="2" cellspacing="0" style="border-collapse: collapse" bordercolor="#111111" id="AutoNumber1">
   <tr><td><pre>
  \end{rawhtml} 
\input{htmlcode/6.6.txt}
 \begin{rawhtml} 
   </pre></td></tr>
  </table>
\end{rawhtml} 
\end{htmlonly}

The return value of the join block is received by 
the join block reception handler. Every join block reception handler
must have only one argument with the type of the object array. 
The following example illustrates how to access the join block return 
values through tokens.  
In lines 16-20, the message {\tt multiply} will not be processed 
until the three messages {\tt add(2,3)}, {\tt add(3,4)}, and {\tt add(2,4)} 
are processed. The token passed to {\tt multiply} is an 
array of {\tt Integers} generated by the three {\tt adds} messages. 
The message handler {\tt multiply(Object numbers[])} in lines 3-7 
is the join block reception handler.
%begin{latexonly} 
{\singlespace
\lstinputlisting[]{code/JoinContinuation.salsa.txt}
}
%end{latexonly} 
\begin{htmlonly}
 \begin{rawhtml} 
  <table border="1" cellpadding="2" cellspacing="0" style="border-collapse: collapse" bordercolor="#111111" id="AutoNumber1">
   <tr><td><pre>
  \end{rawhtml} 
\input{htmlcode/JoinContinuation.salsa.txt}
 \begin{rawhtml} 
   </pre></td></tr>
  </table>
\end{rawhtml} 
\end{htmlonly}

\section{Message Properties}

SALSA provides three message properties that could 
be used with message sending: {\tt priority}, {\tt delay}, 
{\tt waitfor}, and {\tt delayWaitfor}. The syntax used to assign to a message a 
given property is the following, where 
{\tt {\textless}property name{\textgreater}} can be either {\tt priority}, 
{\tt delay}, {\tt waitfor}, and {\tt delayWaitfor}:

\textbf{actor{\textless}-myMessage:{\textless}property name{\textgreater}}


\subsection{Property: {\tt priority}}
The {\tt priority} property is used to send a message with high priority.
This is achieved by placing the message at the beginning 
of the actor's message box.
For instance, the following statement will result in sending the 
message {\tt migrate} to the actor, {\tt book}, with the 
highest property. 
%begin{latexonly} 
{\singlespace
\lstinputlisting[]{code/6.7.txt}
}
%end{latexonly} 
\begin{htmlonly}
 \begin{rawhtml} 
  <table border="1" cellpadding="2" cellspacing="0" style="border-collapse: collapse" bordercolor="#111111" id="AutoNumber1">
   <tr><td><pre>
  \end{rawhtml} 
\input{htmlcode/6.7.txt}
 \begin{rawhtml} 
   </pre></td></tr>
  </table>
\end{rawhtml} 
\end{htmlonly}

Let us assume that the local host is overloaded ,the message box of Actor {\tt book} is full, and
the remote host to migrate has extra computing power. 
Using the {\tt priority} property by attaching it to the {\tt migrate} message may improve the performance.

\subsection{Property: {\tt delay}}
The {\tt delay} property is used to send a message with a given delay. 
It takes as arguments the delay duration in milliseconds. 
The property is usually used as a loose timer.
For instance,
the following message {\tt awaken} will be sent to the 
actor, {\tt book}, after a delay of 1s.
%begin{latexonly} 
{\singlespace
\lstinputlisting[]{code/6.8.txt}
}
%end{latexonly} 
\begin{htmlonly}
 \begin{rawhtml} 
  <table border="1" cellpadding="2" cellspacing="0" style="border-collapse: collapse" bordercolor="#111111" id="AutoNumber1">
   <tr><td><pre>
  \end{rawhtml} 
\input{htmlcode/6.8.txt}
 \begin{rawhtml} 
   </pre></td></tr>
  </table>
\end{rawhtml} 
\end{htmlonly}

\subsection{Property: {\tt waitfor}}
The {\tt waitfor} property is used to wait for the reception of a token 
before sending a message. The property can add variable continuation restrictions dynamically, 
enabling a flexible and-barrier for concurrent execution.
The following example shows that the message {\tt compare(b)} can be delivered to Actor {\tt a} if
Actors {\tt a} and {\tt b} have migrated to the same theater:  
%begin{latexonly} 
{\singlespace
\lstinputlisting[]{code/6.9.txt}
}
%end{latexonly} 
\begin{htmlonly}
 \begin{rawhtml} 
  <table border="1" cellpadding="2" cellspacing="0" style="border-collapse: collapse" bordercolor="#111111" id="AutoNumber1">
   <tr><td><pre>
  \end{rawhtml} 
\input{htmlcode/6.9.txt}
 \begin{rawhtml} 
   </pre></td></tr>
  </table>
\end{rawhtml} 
\end{htmlonly}

\subsection{Property: {\tt delayWaitfor}}
SALSA $\salsaversion$ does not support multiple properties.
{\tt delayWaitfor} is a temporary solution to support
{\tt delay} and {\tt waitfor} in the same message.
The {\tt delayWaitfor} property takes the first argument as the delay duration in milliseconds,
and the remainder as tokens. For instance,
the message {\tt compare(b)} can be delivered to Actor {\tt a} if
Actors {\tt a} and {\tt b} have migrated to the same theater and 
after a delay of 1 second:  

%begin{latexonly} 
{\singlespace
\lstinputlisting[]{code/6.10.txt}
}
%end{latexonly} 
\begin{htmlonly}
 \begin{rawhtml} 
  <table border="1" cellpadding="2" cellspacing="0" style="border-collapse: collapse" bordercolor="#111111" id="AutoNumber1">
   <tr><td><pre>
  \end{rawhtml} 
\input{htmlcode/6.10.txt}
 \begin{rawhtml} 
   </pre></td></tr>
  </table>
\end{rawhtml} 
\end{htmlonly}
